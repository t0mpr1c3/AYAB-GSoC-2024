\documentclass{article}
\usepackage{hyperref}
\usepackage{xcolor}
\usepackage{graphicx} % Required for inserting images

\title{AYAB Google Summer of Code 2024}
\author{\color{red}{\textbf{Application Deadline: Feb 6, 2024}}}

\begin{document}

\maketitle

\section{Mentoring Organisation}

\href{https://ayab-knitting.com/}{AYAB} (All Yarns Are Beautiful) is an open-source software and electronics project that enables a user to design and knit garments using a low-cost home knitting machine. 

AYAB distributes a desktop application that enables the user to knit a multicolor fabric design from a picture or photograph. The software governs an Arduino-based controller that interfaces with Brother 900-series electronic knitting machines. AYAB also publishes the firmware and hardware specifications for the controller. The source code is curated in a mature \href{https://github.com/AllYarnsAreBeautiful}{Github organisation} with 10 repositories and 12 contributing members.

The project was conceived by Christian Obersteiner and Andreas M{\"u}ller in 2013 and has since gained contributors from Belgium, Germany, Poland, Sweden, the UK, and the USA. Theare are active users in at least 15 countries from Europe, Asia, the Americas, and Oceania. The Facebook and Discord groups number 1,552 and 70 members respectively, and there are 402 group members on the fiber craft social media platform Ravelry.

AYAB workshops are conducted as part of the annual \href{http://etextilespringbreak.org/}{eTextile Spring Break}, and the technology is featured in the forthcoming documentary film, \href{https://www.thedomesticmachine.com/}{\textit{The Domestic Machine}.}



\section{Ideas List}

{\itshape
[This is the most important part of the application. Every potential project needs a mentor: please nominate yourself if you would be interested in doing so, and have the time to commit to the role.

The primary focus of projects should probably be the desktop software, which we are planning to rewrite as a webapp. Documentation, and in particular the AYAB website, is another area where contributions would be valuable. Anything that would require a contributor to have access to a knitting machine and/or an AYAB interface could be problematic.

Each project on the Ideas list should include: a) a project title/description b) more detailed description of the project (2-5 sentences) c) expected outcomes d) skills required/preferred e) possible mentors f) expected size of project: 90, 175 or 350 hours.]
}

\begin{itemize}
\item \textbf{Prototype web app for AYAB UI} (175 or 350 hours). 

\textit{Goal.} Redesign the user interface and prototype the replacement in Javascript.

\textit{Description.} Currently the AYAB desktop app, written in PyQt5, is hard to maintain across different operating systems and cannot be operated from a smart phone. AYAB needs contributors who can help retool the UI as a web application. 

A stretch goal would be to extend the functionality of the UI to include stitch-by-stitch editing of the pattern, a feature that has frequently been requested by users.

This a large project whose goals are central to the future of AYAB. The skills learned as part of this project have wide application.

\textit{Skills required.} Some knowledge of Python and basic familiarity with HTML. Previous experience with Javascript is preferred.

\textit{Category.} Core project. 

\textit{Mentors,} Tom, ???.


\item \textbf{Redesign and extend AYAB website} (90 hours). 

\textit{Goal.} Build webpages that document the AYAB project. 

\textit{Description.} The AYAB website has not kept pace with developments in the software, and its compliance with accessibility and W3C standards is unknown. Revising the online material will foster outreach to new users. Better documentation will help existing AYAB users to make best use of the project. Efforts to accommodate non-English speakers would be particularly valuable. 

This project has clear goals and boundaries and is suitable for someone good at written communication who would like to pick up some experience in visual design and web programming.

\textit{Skills required.} The contributor will be mentored in the use of HTML, CSS, Javascript, and Git. Prior experience is not necessary.

\textit{Category.} Low-hanging fruit. 

\textit{Mentors.} Tom, ???.


\item \textbf{Extend AYAB stitch options} (175 hours). 

\textit{Goal.} Reconfigure the internal representation of the AYAB software to allow different stitch types, and change the graphical output to show stitch symbols in the pattern image.

\textit{Description.} AYAB currently implements various different methods for colorwork knitting, but does not allow variations in texture. Allowing different kinds of stitch -- first and foremost Purl stitches, but possibly also increases, decreases, and cables -- would increase the range of designs that can be created and enable AYAB users to make better use of the capabilities of their knitting machines.

\textit{Skills required.} Python or Javascript are preferred.

\textit{Category.} High risk, high reward.

\textit{Mentor.} Tom.


\item \textbf{Increase pattern portability} (175 hours). 

\textit{Goal.} Interface AYAB with additional data sources.

\textit{Description.} Currently, AYAB can upload graphical images and \href{https://www.softbyte.co.uk/designaknit9.htm}{DesignAKnit} pattern files. We aim to expand the range of possible data sources to include formats such as \href{https://textiles-lab.github.io/knitout/knitout.html}{Knitout} and \href{https://stitch-maps.com/about/knitspeak/}{Knitspeak}. This an ongoing research project in which a creative contributor could significantly influence the future direction of AYAB.

\textit{Skills required.} Python or Javascript are preferred.

\textit{Category.} Exploratory; High risk, high reward.

\textit{Mentor.} Tom.
\end{itemize}


\section{Personnel}

{\itshape 
[At least two contributors must serve as organization administrators and/or mentors throughout the entire duration of the program. Include brief resumes of personnel highlighting relevant experience.]
}
\newline

\begin{flushleft}
Org admin: ???
\newline

Mentors: Tom, ???
\end{flushleft}

\href{https://t0mpr1c3.github.io/}{Dr Thomas Price} has been contributing to AYAB since 2021. He has been coding professionally since 2002, and has mentored junior colleagues throughout his career.


\section{GSoC Goals}

{\itshape
\begin{itemize}
\item Inspire developers to begin participating in open source development.
\item Help open source projects identify and bring in new developers.
\item Get more open source code written and released for the benefit of all.
\item Give newer developers more exposure to real-world software development.
\end{itemize}
}


\section{Admin}

{\itshape 
\begin{itemize}
\item The mentoring organisation does not need to be a legal entity, although it will receive money from Google which will presumably need to go through a US bank account.
\item The main part of the program runs from the end of May until the beginning of September.
\end{itemize}
}


\section{Links}

\begin{itemize}\item \href{https://developers.google.com/open-source/gsoc/faq}{FAQ}
\item \href{https://developers.google.com/open-source/gsoc/timeline}{Timeline}
\item \href{https://google.github.io/gsocguides/mentor/defining-a-project-ideas-list.html}{What makes a good ideas list}
\item \href{https://developers.google.com/open-source/gsoc/help/responsibilities#mentor_responsibilities}{Mentor responsibilities}
\item \href{https://developers.google.com/open-source/gsoc/help/responsibilities#org_admin_responsibilities}{Org admin responsibilities}
\item \href{https://google.github.io/gsocguides/mentor/}{Mentor Guide}
\end{itemize}


\section{To Do}

{\itshape 
\begin{itemize}
\item OHSWA certification (Tom/Matei)
\end{itemize}
}

\end{document}
