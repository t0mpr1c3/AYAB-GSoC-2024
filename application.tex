\documentclass{article}
\usepackage{hyperref}
\usepackage{graphicx} % Required for inserting images

\title{AYAB Google Summer of Code 2024}
\author{Application deadline Feb 6, 2024}

\begin{document}

\maketitle

\section{Mentoring Organisation}

\href{https://ayab-knitting.com/}{AYAB} (All Yarns Are Beautiful) is an open-source software and electronics project that enables a user to design and knit garments using a low-cost home knitting machine. 

AYAB distributes a desktop application that enables the user to knit a multicolor fabric design from a picture or photograph. The software governs an Arduino-based controller that interfaces with Brother 900-series electronic knitting machines. AYAB also publishes the firmware and hardware designs for the controller.

The project was conceived by Christian Obersteiner and Andreas M{\"u}ller in 2013 and has since gained contributors from Belgium, Germany, Poland, Sweden, the UK, and the USA. Theare are active users in at least 15 countries from Europe, Asia, the Americas, and Oceania. The Facebook and Discord groups number 1,552 and 70 members respectively, and there are 402 group members on the fiber craft social media platform Ravelry.

The source code is curated using a mature \href{https://github.com/AllYarnsAreBeautiful}{Github organisation} that features 12 contributing members and 10 repositories.

\section{Ideas List}

{\itshape
This is the most important part of the application. Every potential project needs a mentor: please nominate yourself if you would be interested in doing so, and have the time to commit to the role.

The primary focus of projects should probably be the desktop software, which we are planning to rewrite as a webapp. Documentation, and in particular the AYAB website, is another area where contributions would be valuable. Anything that would require a contributor to have access to a knitting machine and/or an AYAB interface will be problematic.

Each project on the Ideas list should include: a) a project title/description b) more detailed description of the project (2-5 sentences) c) expected outcomes d) skills required/preferred e) possible mentors f) expected size of project (90, 175 or 350 hours.)
}

\begin{itemize}
\item \textbf{Recode the AYAB UI as a web app} (175 or 350 hours). 

\textit{Goal.} {\itshape To be defined. We need to describe specific modules of work that a GSoC contributor could have ownership of.}

\textit{Description.} Currently the AYAB desktop app, written in PyQt5, is hard to maintain across different operating systems and cannot be operated from a smart phone. AYAB needs contributors who can help retool the UI as a webapp. This a large project whose goals are central to the future of AYAB. The skills learned as part of this project have wide application.

\textit{Skills required.} Some knowledge of Python and basic familiarity with HTML. Previous experience with Javascript and Git preferred.

\textit{Category.} Core project. 

\textit{Mentors,} Tom, ???.


\item \textbf{Retool AYAB website} (90 hours). 

\textit{Goal.} Build webpages to better document the AYAB project. 

\textit{Description.} New online material will foster accessibility by helping outreach to new users. Better documentation will help existing AYAB users to make best use of the project. Efforts to accommodate non-English speakers would be particularly valuable.

\textit{Skills required.} The project will involve HTML, CSS, Javascript, and Git. Prior experience is not necessary.

\textit{Category.} Low-hanging fruit. 

\textit{Mentors.} Tom, ???.


\item \textbf{Exploratory projects} (175 hours). 

\textit{Description.} Possibilities include: (a) Change the internal representation and graphical output of the AYAB software to allow alternative stitch types. This will enable AYAB to use more of the capabilities of knitting machine equipped with a ribber, Garter carriage, and/or Lace carriage. (b) Interface AYAB with alternative data sources such as Knitout and Knitspeak formats. These are ongoing research projects in which a creative contributor could make important directions to the future of the project.

\textit{Category.} High risk, high reward.

\textit{Mentor.} Tom.
\end{itemize}


\section{Personnel}

\begin{flushleft}
Org admin: ???

Mentors: Tom, ???
\end{flushleft}

{\itshape 
At least two contributors must serve as organization administrators and/or mentors throughout the entire duration of the program.

Include brief resumes of personnel highlighting relevant experience.
}


\section{GSoC Goals}

{\itshape
\begin{itemize}
\item Inspire developers to begin participating in open source development.
\item Help open source projects identify and bring in new developers.
\item Get more open source code written and released for the benefit of all.
\item Give newer developers more exposure to real-world software development.
\end{itemize}
}


\section{Admin}

{\itshape 
\begin{itemize}
\item The mentoring organisation does not need to be a legal entity, although it will receive money from Google which will presumably need to go through a US bank account.
\item The main part of the program runs from the end of May until the beginning of September.
\end{itemize}
}

\section{Links}

\begin{itemize}\item \href{https://developers.google.com/open-source/gsoc/faq}{FAQ}
\item \href{https://developers.google.com/open-source/gsoc/timeline}{Timeline}
\item \href{https://google.github.io/gsocguides/mentor/defining-a-project-ideas-list.html}{What makes a good ideas list}
\item \href{https://developers.google.com/open-source/gsoc/help/responsibilities#mentor_responsibilities}{Mentor responsibilities}
\item \href{https://developers.google.com/open-source/gsoc/help/responsibilities#org_admin_responsibilities}{Org admin responsibilities}
\item \href{https://google.github.io/gsocguides/mentor/}{Mentor Guide}
\end{itemize}

\end{document}
