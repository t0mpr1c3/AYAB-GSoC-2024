\documentclass{article}
\usepackage{hyperref}
\usepackage{graphicx} % Required for inserting images

\title{AYAB Google Summer of Code 2024}
\author{AYAB development team}
\date{December 2023}

\begin{document}

\maketitle

\section{Ideas List}

This is the most important part of the application. Every potential project needs a mentor: please nominate yourself if you would be interested in doing so, and have the time to commit to the role.

The primary focus of projects should probably be the desktop software, which we are planning to rewrite as a webapp. Documentation, and in particular the AYAB website, is another area where contributions would be valuable. Anything that would require a contributor to have access to a knitting machine and/or an AYAB interface will be problematic.

\begin{itemize}
\item Core project (175 or 350 hours). Recode the AYAB UI as a web app. Currently the AYAB desktop app is hard to maintain across different operating systems, and cannot be operated from a smart phone. AYAB needs contributors who can help retool the UI as a webapp. This a large project whose goals are central to the future of AYAB. We need to define specific modules of work that a GSoC contributor could have ownership of. The skills learned as part of this project Mentors: Tom, ???.
\item Low-hanging fruit (90 hours). Build webpages to better document the AYAB project. New online material will foster accessibility by helping efforts to reach new users. Better documentation will help existing AYAB users to make best use of the project. This might be especially suitable for a contributor with language skills who can assist with non-English content: AYAB is a world-wide project, but users without English language skills are currently underserved. Mentors: Tom, ???.
\item Exploratory projects (175 hours). Possibilities include: (a) Change the internal representation and graphical output of the AYAB software to allow alternative stitch types. This will enable AYAB to use more of the capabilities of knitting machine equipped with a ribber, Garter carriage, and/or Lace carriage. (b) Interface AYAB with alternative data sources such as Knitout and Knitspeak formats. These are ongoing research projects in which a creative contributor could make important directions to the future of the project. Mentor: Tom.
\end{itemize}

\section{Personel}

\begin{flushleft}
Org admin: ???

Mentors: Tom, ???
\end{flushleft}

\section{GSoC Goals}

\begin{itemize}
\item Inspire developers to begin participating in open source development.
\item Help open source projects identify and bring in new developers.
\item Get more open source code written and released for the benefit of all.
\item Give newer developers more exposure to real-world software development.
\end{itemize}

\section{Basic Admin}

\begin{itemize}
\item Feb 6, 2024: Mentoring organization application deadline
\item AYAB qualifies as a mentoring organisation because it is an open source software project.
\item The mentoring organisation does not need to be a legal entity, although it will receive money from Google which will presumably need to go through a US bank account.
\item At least two contributors must serve as organization administrators and/or mentors throughout the entire duration of the program.
\item The main part of the program runs from the end of May until the beginning of September.
\end{itemize}

\section{Links}

\begin{itemize}\item \href{https://developers.google.com/open-source/gsoc/faq}{FAQ}
\item \href{https://developers.google.com/open-source/gsoc/timeline}{Timeline}
\item \href{https://google.github.io/gsocguides/mentor/defining-a-project-ideas-list.html}{What makes a good ideas list}
\item \href{https://developers.google.com/open-source/gsoc/help/responsibilities#mentor_responsibilities}{Mentor responsibilities}
\item \href{https://developers.google.com/open-source/gsoc/help/responsibilities#org_admin_responsibilities}{Org admin responsibilities}
\item \href{https://google.github.io/gsocguides/mentor/}{Mentor Guide}
\end{itemize}

\end{document}
