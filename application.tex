
\documentclass{article}
\usepackage{hyperref}
\usepackage{xcolor}
\usepackage{graphicx} % Required for inserting images
\usepackage{parskip}
\setlength{\parskip}{10pt}

\title{AYAB Google Summer of Code 2024}
\author{\color{red}{\textbf{Application Deadline: Feb 6, 2024}}}

\begin{document}

\maketitle

\section{Mentoring Organisation}

\subsection{Mission}

\href{https://ayab-knitting.com/}{AYAB} (All Yarns Are Beautiful) is an open-source software and electronics project that enables users to design and knit garments using a low-cost home knitting machine.

\subsection{Product}

AYAB distributes a desktop application that enables the user to knit a multicolor fabric design from a picture or photograph. The software governs an Arduino-based controller that interfaces with Brother 900-series electronic knitting machines. AYAB also publishes the firmware and hardware specifications for the controller. The source code is curated in a mature \href{https://github.com/AllYarnsAreBeautiful}{Github organisation} with 10 repositories and 12 contributing members.

\subsection{Impact}

The project was conceived by Christian Obersteiner and Andreas M{\"u}ller in 2013 and has since gained contributors from Belgium, Germany, Poland, Sweden, the UK, and the USA. There are are active users in at least 15 countries from Europe, Asia, the Americas, and Oceania. The Facebook and Discord groups number 1,552 and 70 members respectively, and there are 402 group members on the fiber craft social media platform Ravelry.

AYAB workshops are conducted as part of the annual \href{http://etextilespringbreak.org/}{eTextile Spring Break}, and the technology has been featured in numerous media, including MAKE Magazine and the forthcoming documentary film, \href{https://www.thedomesticmachine.com/}{\textit{The Domestic Machine}.}

The AYAB project has already participated successfully in GSoC 2014, 2015 and 2016 as part of the FOSSASIA organization.

\section{Project Proposals}

{\itshape
[This is the most important part of the application. Every potential project needs a mentor: please nominate yourself if you would be interested in doing so, and have the time to commit to the role.

The primary focus of projects should probably be the desktop software, which we are planning to rewrite as a webapp. Documentation, and in particular the AYAB website, is another area where contributions would be valuable. Anything that would require a contributor to have access to a knitting machine and/or an AYAB interface could be problematic.

Each project on the Ideas list should include: a) a project title/description b) more detailed description of the project (2-5 sentences) c) expected outcomes d) skills required/preferred e) possible mentors f) expected size of project: 90, 175 or 350 hours.]
}

\subsection{Prototype web app for AYAB UI}

\textit{Goal.} Redesign the user interface and prototype a new client.

\textit{Description.} Currently the AYAB desktop app, written in PyQt5, is hard to maintain across different operating systems and cannot be operated from a smart phone. AYAB has committed to re-implementing the UI as a web application.

The web app will be prototyped as a standalone client that communicates with the AYAB interface over USB. The eventual goal is to host the backend on a ESP32 microcontroller embedded in the AYAB interface itself. A stretch goal is to extend the functionality of the UI to include stitch-by-stitch editing of the pattern, a feature that has frequently been requested by users.

The precise scope of the work will be negotiated with the student, taking into account their availability and skill set.

The goals of this project are central to the future of AYAB. The skills learned as part of this project have wide application.

\textit{Skills required.} Some knowledge of Python and basic familiarity with HTML. Previous experience with Javascript and UI technology is preferred.

\textit{Category.} Core project. (175 or 350 hours).

\textit{Mentors,} Tom, ???.


\subsection{Develop AYAB website} 

\textit{Goal.} Design and build a website with wiki or social media features that enables users to document the AYAB project. 

\textit{Description.} The AYAB website has not kept pace with developments in the software, and its compliance with accessibility and W3C standards is unknown. Better documentation will help existing AYAB users to make best use of the project. Revising the online material will foster outreach to new users. There is, for example, a good deal of relevant video material in the public domain that could be incorporated into the educational content.

Currently the source of the documentation is hosted in Github, which presents a high barrier to participation for many AYAB users. A website with wiki or blogging features will allow less technically-minded knitters to contribute to the ecosystem. Outreach to AYAB users on social media has shown that efforts to accommodate non-English speakers may be particularly valuable.

\textit{Skills required.} The contributor will be mentored in the use of HTML, CSS, Javascript, and Git. Prior experience is not necessary.

\textit{Category.} Low-hanging fruit. (90 or 175 hours).

\textit{Mentors.} Tom, ???.


\subsection{Extend AYAB stitch options} 

\textit{Goal.} Reconfigure the internal representation of the AYAB software to allow different stitch types, and change the graphical output to show stitch symbols in the pattern image.

\textit{Description.} AYAB currently implements various different methods for colorwork knitting, but does not allow variations in texture. Allowing different kinds of stitch -- first and foremost Purl stitches, but possibly also increases, decreases, and cables -- would increase the range of designs that can be created and enable AYAB users to make better use of the capabilities of their knitting machines.

\textit{Skills required.} Python or Javascript are preferred.

\textit{Category.} High risk, high reward. (175 hours). 

\textit{Mentor.} Tom.


\subsection{Increase pattern portability} 

\textit{Goal.} AYAB with additional data sources.

\textit{Description.} Currently, AYAB can upload graphical images and \href{https://www.softbyte.co.uk/designaknit9.htm}{DesignAKnit} pattern files. We aim to expand the range of possible data sources to include formats such as \href{https://textiles-lab.github.io/knitout/knitout.html}{Knitout} and \href{https://stitch-maps.com/about/knitspeak/}{Knitspeak}. This an ongoing research project in which a creative contributor could significantly influence the future direction of AYAB.

\textit{Skills required.} Python or Javascript are preferred.

\textit{Category.} Exploratory; High risk, high reward. (175 hours). 

\textit{Mentor.} Tom.


\section{Administration}

\subsection{Communications and updates}

The student and mentor will communicate frequently to develop the project in an efficient manner and
maintain a structured software development approach. Specifically:

The student will send an email or chat message each morning answering the following three questions:
1) What did you do yesterday? What was the outcome?
2) What do you intend to do? What is your intended outcome?
3) Is there anything that might prevent you from achieving this goal? How can others help
you?

A backlog will be maintained in a dedicated \href{http://www.redmine.org/}{Redmine} tracking system. The GSoC student is
required to fill in updates during the day.

The student will schedule weekly Zoom meetings with their mentor. They may also communicate \textit{ad hoc}. The mentor will email the organisational admin on a weekly basis to summarize recent developments.

\subsection{Deliverables}
After each iteration, the following artifacts will be submitted by the student to a Git repository: (a) source code, (b) \href{https://jsdoc.app/}{JSDoc} or \href{https://www.sphinx-doc.org/en/master/}{Sphinx} code annotations, (c) installation instructions in Markdown or HTML.

\subsection{Constraints}
The developer can make use of software components as necessary for the project, provided that they are licensed under an Open Source license and can be obtained and redistributed freely.

\subsection{Timelines}
The mentor will conduct an interim evaluation in the first week of July. Final evaluations will take place in the week of August 19-26. The organisational administrator is responsible for submitting the evaluations.


\section{Personnel}

{\itshape 
[At least two contributors must serve as organization administrators and/or mentors throughout the entire duration of the program. Include brief resumes of personnel highlighting relevant experience.]
}

\begin{itemize}
\item \textbf{Organisational admin:} Chris

A professional software engineer and project manager, Chris co-founded the AYAB project in 2013 and has been an active participant ever since. He has contributed to previous rounds of GSoC as a member of the open-source FOSSASIA organization.

\item \textbf{Mentor:} Tom, ???

\href{https://t0mpr1c3.github.io/}{Dr Thomas Price} has been contributing to AYAB since 2021. He has been coding professionally since 2002, and has mentored junior colleagues throughout his career.

\item \textbf{Additional Mentors:} ???
\end{itemize}

\newpage
\appendix 
\LARGE \textbf{Notes} \normalsize

\section{GSoC Goals}

{\itshape
\begin{itemize}
\item Inspire developers to begin participating in open source development.
\item Help open source projects identify and bring in new developers.
\item Get more open source code written and released for the benefit of all.
\item Give newer developers more exposure to real-world software development.
\end{itemize}
}


\section{Administrative requirements}

{\itshape 
\begin{itemize}
\item The mentoring organisation does not need to be a legal entity, although it will receive money from Google which will presumably need to go through a US bank account.
\item The main part of the program runs from the end of May until the beginning of September.
\end{itemize}
}


\section{Links}

\begin{itemize}\item \href{https://developers.google.com/open-source/gsoc/faq}{FAQ}
\item \href{https://developers.google.com/open-source/gsoc/timeline}{Timeline}
\item \href{https://google.github.io/gsocguides/mentor/defining-a-project-ideas-list.html}{What makes a good ideas list}
\item \href{https://developers.google.com/open-source/gsoc/help/responsibilities#mentor_responsibilities}{Mentor responsibilities}
\item \href{https://developers.google.com/open-source/gsoc/help/responsibilities#org_admin_responsibilities}{Org admin responsibilities}
\item \href{https://google.github.io/gsocguides/mentor/}{Mentor Guide}
\end{itemize}


\section{To Do}

{\itshape 
\begin{itemize}
\item OHSWA certification (Tom/Matei)
\end{itemize}
}

\end{document}
